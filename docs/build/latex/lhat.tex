%% Generated by Sphinx.
\def\sphinxdocclass{report}
\documentclass[letterpaper,10pt,english]{sphinxmanual}
\ifdefined\pdfpxdimen
   \let\sphinxpxdimen\pdfpxdimen\else\newdimen\sphinxpxdimen
\fi \sphinxpxdimen=.75bp\relax

\PassOptionsToPackage{warn}{textcomp}
\usepackage[utf8]{inputenc}
\ifdefined\DeclareUnicodeCharacter
% support both utf8 and utf8x syntaxes
  \ifdefined\DeclareUnicodeCharacterAsOptional
    \def\sphinxDUC#1{\DeclareUnicodeCharacter{"#1}}
  \else
    \let\sphinxDUC\DeclareUnicodeCharacter
  \fi
  \sphinxDUC{00A0}{\nobreakspace}
  \sphinxDUC{2500}{\sphinxunichar{2500}}
  \sphinxDUC{2502}{\sphinxunichar{2502}}
  \sphinxDUC{2514}{\sphinxunichar{2514}}
  \sphinxDUC{251C}{\sphinxunichar{251C}}
  \sphinxDUC{2572}{\textbackslash}
\fi
\usepackage{cmap}
\usepackage[T1]{fontenc}
\usepackage{amsmath,amssymb,amstext}
\usepackage{babel}



\usepackage{times}
\expandafter\ifx\csname T@LGR\endcsname\relax
\else
% LGR was declared as font encoding
  \substitutefont{LGR}{\rmdefault}{cmr}
  \substitutefont{LGR}{\sfdefault}{cmss}
  \substitutefont{LGR}{\ttdefault}{cmtt}
\fi
\expandafter\ifx\csname T@X2\endcsname\relax
  \expandafter\ifx\csname T@T2A\endcsname\relax
  \else
  % T2A was declared as font encoding
    \substitutefont{T2A}{\rmdefault}{cmr}
    \substitutefont{T2A}{\sfdefault}{cmss}
    \substitutefont{T2A}{\ttdefault}{cmtt}
  \fi
\else
% X2 was declared as font encoding
  \substitutefont{X2}{\rmdefault}{cmr}
  \substitutefont{X2}{\sfdefault}{cmss}
  \substitutefont{X2}{\ttdefault}{cmtt}
\fi


\usepackage[Bjarne]{fncychap}
\usepackage{sphinx}

\fvset{fontsize=\small}
\usepackage{geometry}


% Include hyperref last.
\usepackage{hyperref}
% Fix anchor placement for figures with captions.
\usepackage{hypcap}% it must be loaded after hyperref.
% Set up styles of URL: it should be placed after hyperref.
\urlstyle{same}

\addto\captionsenglish{\renewcommand{\contentsname}{Contents:}}

\usepackage{sphinxmessages}
\setcounter{tocdepth}{1}



\title{LHAT}
\date{Dec 20, 2021}
\release{0.1}
\author{Giorgio Santinelli, Robyn Gwee, Faraz Tehrani}
\newcommand{\sphinxlogo}{\vbox{}}
\renewcommand{\releasename}{Release}
\makeindex
\begin{document}

\ifdefined\shorthandoff
  \ifnum\catcode`\=\string=\active\shorthandoff{=}\fi
  \ifnum\catcode`\"=\active\shorthandoff{"}\fi
\fi

\pagestyle{empty}
\sphinxmaketitle
\pagestyle{plain}
\sphinxtableofcontents
\pagestyle{normal}
\phantomsection\label{\detokenize{index::doc}}


Landslide Hazard Assessment Tool (LHAT) is a rapid assessment tool for landslide hazards around the world.

This online documentation describes the tool, its usage and the model choices that can be used to derive landslide susceptibility maps.
For further enquiries, please approach the following developers: Giorgio Santinelli (\sphinxhref{mailto:giorgio.santinelli@deltares.nl}{giorgio.santinelli@deltares.nl}) and Robyn Gwee (\sphinxhref{mailto:robyn.gwee@deltares.nl}{robyn.gwee@deltares.nl})


\chapter{Installing LHAT}
\label{\detokenize{index:installing-lhat}}
Clone the LHAT repository locally from \sphinxurl{https://github.com/openearth/lhat}

\begin{sphinxVerbatim}[commandchars=\\\{\}]
\PYGZgt{}\PYGZgt{} git clone https://github.com/openearth/lhat.git
\end{sphinxVerbatim}

Navigate to the directory where you cloned the repository and create a conda environment from the yml file.

\begin{sphinxVerbatim}[commandchars=\\\{\}]
\PYGZgt{}\PYGZgt{} conda env create \PYGZhy{}f environment.yml
\end{sphinxVerbatim}

An example file has been made to showcase how LHAT can be parameterised and run. See: example.py
Or alternatively, run the script as follows:

\begin{sphinxVerbatim}[commandchars=\\\{\}]
\PYGZgt{}\PYGZgt{} python example.py
\end{sphinxVerbatim}
\phantomsection\label{\detokenize{index:module-lhat}}\index{module@\spxentry{module}!lhat@\spxentry{lhat}}\index{lhat@\spxentry{lhat}!module@\spxentry{module}}
Landslide Hazard Assessment Tool (LHAT)
Copyright (C) 2021 Robyn Gwee, Giorgio Santinelli,

This program is free software: you can redistribute it and/or modify
it under the terms of the GNU General Public License as published by
the Free Software Foundation, either version 3 of the License, or
(at your option) any later version.

This program is distributed in the hope that it will be useful,
but WITHOUT ANY WARRANTY; without even the implied warranty of
MERCHANTABILITY or FITNESS FOR A PARTICULAR PURPOSE.  See the
GNU General Public License for more details.

You should have received a copy of the GNU General Public License
along with this program.  If not, see \textless{}\sphinxurl{https://www.gnu.org/licenses/}\textgreater{}.


\section{lhat}
\label{\detokenize{modules:lhat}}\label{\detokenize{modules::doc}}

\subsection{lhat package}
\label{\detokenize{lhat:lhat-package}}\label{\detokenize{lhat::doc}}

\subsubsection{lhat.IO: Data operations on input and output data}
\label{\detokenize{lhat:module-lhat.IO}}\label{\detokenize{lhat:lhat-io-data-operations-on-input-and-output-data}}\index{module@\spxentry{module}!lhat.IO@\spxentry{lhat.IO}}\index{lhat.IO@\spxentry{lhat.IO}!module@\spxentry{module}}\index{inputData (class in lhat.IO)@\spxentry{inputData}\spxextra{class in lhat.IO}}

\begin{fulllineitems}
\phantomsection\label{\detokenize{lhat:lhat.IO.inputData}}\pysiglinewithargsret{\sphinxbfcode{\sphinxupquote{class }}\sphinxcode{\sphinxupquote{lhat.IO.}}\sphinxbfcode{\sphinxupquote{inputData}}}{\emph{\DUrole{n}{name}}, \emph{\DUrole{n}{path}}, \emph{\DUrole{n}{dtype}}}{}
Bases: \sphinxcode{\sphinxupquote{object}}
\begin{quote}\begin{description}
\item[{Parameters}] \leavevmode\begin{itemize}
\item {} 
\sphinxstyleliteralstrong{\sphinxupquote{name}} (\sphinxstyleliteralemphasis{\sphinxupquote{str}}) \textendash{} Name of input data. Eg. ‘road’, ‘dem’, ‘vegetation’, etc..

\item {} 
\sphinxstyleliteralstrong{\sphinxupquote{path}} (\sphinxstyleliteralemphasis{\sphinxupquote{str}}) \textendash{} Absolute or relative path to where the input data is located

\item {} 
\sphinxstyleliteralstrong{\sphinxupquote{dtype}} (\sphinxstyleliteralemphasis{\sphinxupquote{str}}\sphinxstyleliteralemphasis{\sphinxupquote{ (}}\sphinxstyleliteralemphasis{\sphinxupquote{\textquotesingle{}numerical\textquotesingle{}}}\sphinxstyleliteralemphasis{\sphinxupquote{ or }}\sphinxstyleliteralemphasis{\sphinxupquote{\textquotesingle{}categorical\textquotesingle{}}}\sphinxstyleliteralemphasis{\sphinxupquote{)}}) \textendash{} The data type is important to define as categorical and numerical datasets
are treated differently in the IO module.

\end{itemize}

\item[{Returns}] \leavevmode
An object containing attributes of input data including name, filepath
and data type

\item[{Return type}] \leavevmode
{\hyperref[\detokenize{lhat:lhat.IO.inputData}]{\sphinxcrossref{\sphinxcode{\sphinxupquote{inputData}}}}}

\end{description}\end{quote}

\end{fulllineitems}

\index{inputs (class in lhat.IO)@\spxentry{inputs}\spxextra{class in lhat.IO}}

\begin{fulllineitems}
\phantomsection\label{\detokenize{lhat:lhat.IO.inputs}}\pysiglinewithargsret{\sphinxbfcode{\sphinxupquote{class }}\sphinxcode{\sphinxupquote{lhat.IO.}}\sphinxbfcode{\sphinxupquote{inputs}}}{\emph{\DUrole{n}{project\_name}\DUrole{p}{:} \DUrole{n}{str}}, \emph{\DUrole{n}{crs}\DUrole{p}{:} \DUrole{n}{str}}, \emph{\DUrole{n}{landslide\_points}\DUrole{p}{:} \DUrole{n}{str}}, \emph{\DUrole{n}{random\_state}\DUrole{p}{:} \DUrole{n}{int}}, \emph{\DUrole{n}{bbox}\DUrole{p}{:} \DUrole{n}{list}}, \emph{\DUrole{n}{inputs}\DUrole{p}{:} \DUrole{n}{dict}}, \emph{\DUrole{n}{no\_data}\DUrole{p}{:} \DUrole{n}{Optional\DUrole{p}{{[}}list\DUrole{p}{{]}}} \DUrole{o}{=} \DUrole{default_value}{None}}, \emph{\DUrole{n}{pixel\_size}\DUrole{p}{:} \DUrole{n}{Optional\DUrole{p}{{[}}int\DUrole{p}{{]}}} \DUrole{o}{=} \DUrole{default_value}{None}}, \emph{\DUrole{n}{kernel\_size}\DUrole{p}{:} \DUrole{n}{int} \DUrole{o}{=} \DUrole{default_value}{3}}}{}
Bases: \sphinxcode{\sphinxupquote{object}}

The inputs class initialises a project and saves all relevant data within
the project folder.

Initialize project with various inputs.
\begin{quote}\begin{description}
\item[{Parameters}] \leavevmode\begin{itemize}
\item {} 
\sphinxstyleliteralstrong{\sphinxupquote{project\_name}} (\sphinxstyleliteralemphasis{\sphinxupquote{str}}) \textendash{} Name of project eg. ‘Jamaica’

\item {} 
\sphinxstyleliteralstrong{\sphinxupquote{crs}} (\sphinxstyleliteralemphasis{\sphinxupquote{str}}) \textendash{} CRS to reproject all input data to and for model result

\item {} 
\sphinxstyleliteralstrong{\sphinxupquote{landslide\_points}} (\sphinxstyleliteralemphasis{\sphinxupquote{str}}) \textendash{} Path to landslide points file (takes geoJSON or shapefile)

\item {} 
\sphinxstyleliteralstrong{\sphinxupquote{bbox}} (\sphinxstyleliteralemphasis{\sphinxupquote{list}}) \textendash{} Bounding box. Required for downloading or clipping online datasets.
Takes in a list of coordinates in WGS84, or a shapefile, or a geoJSON.
Example: {[}{[}x1,y1{]}, {[}x1,y2{]}, {[}x2,y2{]}, {[}x2,y1{]}, {[}x1,y1{]}{]}

\item {} 
\sphinxstyleliteralstrong{\sphinxupquote{random\_state}} (\sphinxstyleliteralemphasis{\sphinxupquote{int}}) \textendash{} Takes an int of any value to determine a (reproducible) state of randomness.
Determining the random state allows for results to be replicated if needed.

\item {} 
\sphinxstyleliteralstrong{\sphinxupquote{inputs}} (\sphinxstyleliteralemphasis{\sphinxupquote{dict}}) \textendash{} Dictionary of input data pointing to paths or decision to include
it in the model. An example of how to define inputs is provided
in \sphinxcode{\sphinxupquote{example.py}}.

\item {} 
\sphinxstyleliteralstrong{\sphinxupquote{no\_data}} (\sphinxstyleliteralemphasis{\sphinxupquote{list}}) \textendash{} list of potential no\_data values. Preferably, you should define
a constant no\_data value for all your input datasets, so that valid
data will NOT be accidentally masked out. No warranty is provided
for erroneous results if any no data values are a valid value in
another dataset.

\item {} 
\sphinxstyleliteralstrong{\sphinxupquote{pixel\_size}} (\sphinxstyleliteralemphasis{\sphinxupquote{int}}) \textendash{} Resolution of pixel size. WARNING \sphinxhyphen{} pixel size is only relevant
for datasets obtained online. The pixel size will therefore
dictate the resolution that the (relevant) input data will be in.

\item {} 
\sphinxstyleliteralstrong{\sphinxupquote{kernel\_size}} (\sphinxstyleliteralemphasis{\sphinxupquote{int = 3}}) \textendash{} To account for potential uncertainty in landslide\sphinxhyphen{}striken areas,
a default {[}3 x 3{]} window of pixels around landslide points
is classed as ‘landslide’. The user can define a larger kernel size
of ODD numbers eg. 3, 5, 7, 9 etc, depending on the resolution
of their input datasets.

\end{itemize}

\item[{Returns}] \leavevmode
Object is returned containing columns of input data as defined by
the user {[}x{]} as well as another pandas.DataFrame object of classes.
Each row represents a pixel index in the stack of input datasets.

\item[{Return type}] \leavevmode
{\color{red}\bfseries{}:object:\textasciigrave{}pandas.DataFrame\textasciigrave{}}

\end{description}\end{quote}
\index{generate\_xy() (lhat.IO.inputs method)@\spxentry{generate\_xy()}\spxextra{lhat.IO.inputs method}}

\begin{fulllineitems}
\phantomsection\label{\detokenize{lhat:lhat.IO.inputs.generate_xy}}\pysiglinewithargsret{\sphinxbfcode{\sphinxupquote{generate\_xy}}}{}{}
Takes the landslide points and selects pixels that overlap with the landslide points
as well as the matrix area around it. Kernel is definable by user.

Returns a dataframe of raster values in those indexes

to\sphinxhyphen{}do: use random\_State attribute for sampling

\end{fulllineitems}

\index{matrix\_window() (lhat.IO.inputs method)@\spxentry{matrix\_window()}\spxextra{lhat.IO.inputs method}}

\begin{fulllineitems}
\phantomsection\label{\detokenize{lhat:lhat.IO.inputs.matrix_window}}\pysiglinewithargsret{\sphinxbfcode{\sphinxupquote{matrix\_window}}}{\emph{\DUrole{n}{x}}, \emph{\DUrole{n}{y}}, \emph{\DUrole{n}{ID}}}{}
kernel size is a default value of 3: can only be an odd number
Generates a dataframe of x and y coordinates

\end{fulllineitems}

\index{proximity2feature() (lhat.IO.inputs method)@\spxentry{proximity2feature()}\spxextra{lhat.IO.inputs method}}

\begin{fulllineitems}
\phantomsection\label{\detokenize{lhat:lhat.IO.inputs.proximity2feature}}\pysiglinewithargsret{\sphinxbfcode{\sphinxupquote{proximity2feature}}}{\emph{\DUrole{n}{rastervec}\DUrole{p}{:} \DUrole{n}{str}}}{}
\end{fulllineitems}

\index{run\_model() (lhat.IO.inputs method)@\spxentry{run\_model()}\spxextra{lhat.IO.inputs method}}

\begin{fulllineitems}
\phantomsection\label{\detokenize{lhat:lhat.IO.inputs.run_model}}\pysiglinewithargsret{\sphinxbfcode{\sphinxupquote{run\_model}}}{\emph{\DUrole{n}{model}\DUrole{p}{:} \DUrole{n}{str}}, \emph{\DUrole{n}{x}}, \emph{\DUrole{n}{y}}, \emph{\DUrole{n}{modelExist}\DUrole{p}{:} \DUrole{n}{bool}}}{}
model choices:  {[}‘SVM’, ‘RF’, ‘LR’{]}. Default: ‘SVM’ (Support Vector Machine)

\end{fulllineitems}

\index{valid\_arrays() (lhat.IO.inputs method)@\spxentry{valid\_arrays()}\spxextra{lhat.IO.inputs method}}

\begin{fulllineitems}
\phantomsection\label{\detokenize{lhat:lhat.IO.inputs.valid_arrays}}\pysiglinewithargsret{\sphinxbfcode{\sphinxupquote{valid\_arrays}}}{}{}
Generates list of valid arrays. A mask is made of only valid arrays
across stack of arrays.

\end{fulllineitems}

\index{vector2raster() (lhat.IO.inputs method)@\spxentry{vector2raster()}\spxextra{lhat.IO.inputs method}}

\begin{fulllineitems}
\phantomsection\label{\detokenize{lhat:lhat.IO.inputs.vector2raster}}\pysiglinewithargsret{\sphinxbfcode{\sphinxupquote{vector2raster}}}{\emph{\DUrole{n}{vectorpath}}}{}
\end{fulllineitems}


\end{fulllineitems}



\subsubsection{lhat.Model: Model parameterisation and grid search}
\label{\detokenize{lhat:module-lhat.Model}}\label{\detokenize{lhat:lhat-model-model-parameterisation-and-grid-search}}\index{module@\spxentry{module}!lhat.Model@\spxentry{lhat.Model}}\index{lhat.Model@\spxentry{lhat.Model}!module@\spxentry{module}}\index{MachineLearning (class in lhat.Model)@\spxentry{MachineLearning}\spxextra{class in lhat.Model}}

\begin{fulllineitems}
\phantomsection\label{\detokenize{lhat:lhat.Model.MachineLearning}}\pysiglinewithargsret{\sphinxbfcode{\sphinxupquote{class }}\sphinxcode{\sphinxupquote{lhat.Model.}}\sphinxbfcode{\sphinxupquote{MachineLearning}}}{\emph{\DUrole{n}{X}}, \emph{\DUrole{n}{y}}, \emph{\DUrole{n}{pathToSavedModel}}, \emph{\DUrole{n}{model\_name}\DUrole{o}{=}\DUrole{default_value}{\textquotesingle{}SVM\textquotesingle{}}}, \emph{\DUrole{n}{modelExist}\DUrole{o}{=}\DUrole{default_value}{False}}}{}
Bases: \sphinxcode{\sphinxupquote{object}}
\index{bestModel (lhat.Model.MachineLearning attribute)@\spxentry{bestModel}\spxextra{lhat.Model.MachineLearning attribute}}

\begin{fulllineitems}
\phantomsection\label{\detokenize{lhat:lhat.Model.MachineLearning.bestModel}}\pysigline{\sphinxbfcode{\sphinxupquote{bestModel}}}
Model name is not correctly specified!

\end{fulllineitems}

\index{evaluateTrainedModel() (lhat.Model.MachineLearning method)@\spxentry{evaluateTrainedModel()}\spxextra{lhat.Model.MachineLearning method}}

\begin{fulllineitems}
\phantomsection\label{\detokenize{lhat:lhat.Model.MachineLearning.evaluateTrainedModel}}\pysiglinewithargsret{\sphinxbfcode{\sphinxupquote{evaluateTrainedModel}}}{\emph{\DUrole{n}{model}}, \emph{\DUrole{n}{X}}, \emph{\DUrole{n}{y}}}{}
\end{fulllineitems}

\index{loadMLModel() (lhat.Model.MachineLearning method)@\spxentry{loadMLModel()}\spxextra{lhat.Model.MachineLearning method}}

\begin{fulllineitems}
\phantomsection\label{\detokenize{lhat:lhat.Model.MachineLearning.loadMLModel}}\pysiglinewithargsret{\sphinxbfcode{\sphinxupquote{loadMLModel}}}{}{}
\end{fulllineitems}

\index{logisticRegression() (lhat.Model.MachineLearning method)@\spxentry{logisticRegression()}\spxextra{lhat.Model.MachineLearning method}}

\begin{fulllineitems}
\phantomsection\label{\detokenize{lhat:lhat.Model.MachineLearning.logisticRegression}}\pysiglinewithargsret{\sphinxbfcode{\sphinxupquote{logisticRegression}}}{}{}
\end{fulllineitems}

\index{predict\_proba() (lhat.Model.MachineLearning method)@\spxentry{predict\_proba()}\spxextra{lhat.Model.MachineLearning method}}

\begin{fulllineitems}
\phantomsection\label{\detokenize{lhat:lhat.Model.MachineLearning.predict_proba}}\pysiglinewithargsret{\sphinxbfcode{\sphinxupquote{predict\_proba}}}{\emph{\DUrole{n}{raster\_stack}}, \emph{\DUrole{n}{estimator}}, \emph{\DUrole{n}{scaler}}, \emph{\DUrole{n}{file\_path}}, \emph{\DUrole{n}{reference}}, \emph{\DUrole{n}{no\_data}}}{}
Apply class probability prediction of a scikit learn model to a RasterStack.
\begin{quote}\begin{description}
\item[{Parameters}] \leavevmode
\sphinxstyleliteralstrong{\sphinxupquote{estimator}} (\sphinxstyleliteralemphasis{\sphinxupquote{estimator object implementing \textquotesingle{}fit\textquotesingle{}}}) \textendash{} The object to use to fit the data.

\end{description}\end{quote}

\end{fulllineitems}

\index{probfun() (lhat.Model.MachineLearning method)@\spxentry{probfun()}\spxextra{lhat.Model.MachineLearning method}}

\begin{fulllineitems}
\phantomsection\label{\detokenize{lhat:lhat.Model.MachineLearning.probfun}}\pysiglinewithargsret{\sphinxbfcode{\sphinxupquote{probfun}}}{\emph{\DUrole{n}{img}}, \emph{\DUrole{n}{estimator}}, \emph{\DUrole{n}{scaler}}}{}
Class probabilities function.
\begin{quote}\begin{description}
\item[{Parameters}] \leavevmode\begin{itemize}
\item {} 
\sphinxstyleliteralstrong{\sphinxupquote{img}} (\sphinxstyleliteralemphasis{\sphinxupquote{tuple}}\sphinxstyleliteralemphasis{\sphinxupquote{ (}}\sphinxstyleliteralemphasis{\sphinxupquote{window}}\sphinxstyleliteralemphasis{\sphinxupquote{, }}\sphinxstyleliteralemphasis{\sphinxupquote{numpy.ndarray}}\sphinxstyleliteralemphasis{\sphinxupquote{)}}) \textendash{} A window object, and a 3d ndarray of raster data with the dimensions in
order of (band, rows, columns).

\item {} 
\sphinxstyleliteralstrong{\sphinxupquote{estimator}} (\sphinxstyleliteralemphasis{\sphinxupquote{estimator object implementing \textquotesingle{}fit\textquotesingle{}}}) \textendash{} The object to use to fit the data.

\end{itemize}

\item[{Returns}] \leavevmode
Multi band raster as a 3d numpy array containing the probabilities
associated with each class. ndarray dimensions are in the order of
(class, row, column).

\item[{Return type}] \leavevmode
numpy.ndarray

\end{description}\end{quote}

\end{fulllineitems}

\index{randomForest() (lhat.Model.MachineLearning method)@\spxentry{randomForest()}\spxextra{lhat.Model.MachineLearning method}}

\begin{fulllineitems}
\phantomsection\label{\detokenize{lhat:lhat.Model.MachineLearning.randomForest}}\pysiglinewithargsret{\sphinxbfcode{\sphinxupquote{randomForest}}}{}{}
\end{fulllineitems}

\index{saveMLModel() (lhat.Model.MachineLearning method)@\spxentry{saveMLModel()}\spxextra{lhat.Model.MachineLearning method}}

\begin{fulllineitems}
\phantomsection\label{\detokenize{lhat:lhat.Model.MachineLearning.saveMLModel}}\pysiglinewithargsret{\sphinxbfcode{\sphinxupquote{saveMLModel}}}{}{}
\end{fulllineitems}

\index{supportVectorMachine() (lhat.Model.MachineLearning method)@\spxentry{supportVectorMachine()}\spxextra{lhat.Model.MachineLearning method}}

\begin{fulllineitems}
\phantomsection\label{\detokenize{lhat:lhat.Model.MachineLearning.supportVectorMachine}}\pysiglinewithargsret{\sphinxbfcode{\sphinxupquote{supportVectorMachine}}}{}{}
\end{fulllineitems}

\index{trainModel() (lhat.Model.MachineLearning method)@\spxentry{trainModel()}\spxextra{lhat.Model.MachineLearning method}}

\begin{fulllineitems}
\phantomsection\label{\detokenize{lhat:lhat.Model.MachineLearning.trainModel}}\pysiglinewithargsret{\sphinxbfcode{\sphinxupquote{trainModel}}}{\emph{\DUrole{n}{baselineModel}}, \emph{\DUrole{n}{modelParameters}}}{}~\begin{quote}\begin{description}
\item[{Parameters}] \leavevmode\begin{itemize}
\item {} 
\sphinxstyleliteralstrong{\sphinxupquote{baselineModel}} \textendash{} This is the model that we use for training: SVC, RF, LR

\item {} 
\sphinxstyleliteralstrong{\sphinxupquote{modelParameters}} \textendash{} This is a set of values for hyper parameters of the model that is used in cross\sphinxhyphen{}validation

\end{itemize}

\item[{Returns}] \leavevmode
the best model

\end{description}\end{quote}

\end{fulllineitems}

\index{trainTestSplit() (lhat.Model.MachineLearning method)@\spxentry{trainTestSplit()}\spxextra{lhat.Model.MachineLearning method}}

\begin{fulllineitems}
\phantomsection\label{\detokenize{lhat:lhat.Model.MachineLearning.trainTestSplit}}\pysiglinewithargsret{\sphinxbfcode{\sphinxupquote{trainTestSplit}}}{}{}
\end{fulllineitems}


\end{fulllineitems}



\chapter{Indices and tables}
\label{\detokenize{index:indices-and-tables}}\begin{itemize}
\item {} 
\DUrole{xref,std,std-ref}{genindex}

\item {} 
\DUrole{xref,std,std-ref}{modindex}

\item {} 
\DUrole{xref,std,std-ref}{search}

\end{itemize}


\renewcommand{\indexname}{Python Module Index}
\begin{sphinxtheindex}
\let\bigletter\sphinxstyleindexlettergroup
\bigletter{l}
\item\relax\sphinxstyleindexentry{lhat.IO}\sphinxstyleindexpageref{lhat:\detokenize{module-lhat.IO}}
\item\relax\sphinxstyleindexentry{lhat.Model}\sphinxstyleindexpageref{lhat:\detokenize{module-lhat.Model}}
\end{sphinxtheindex}

\renewcommand{\indexname}{Index}
\printindex
\end{document}